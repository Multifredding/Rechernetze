\documentclass[a4paper]{article}
\usepackage[utf8]{inputenc}
\usepackage{lmodern}
\usepackage[ngerman]{babel}
\usepackage[top=3cm, bottom=3cm, left=2.5cm, right=2.5cm]{geometry}
\usepackage{titletoc}
\usepackage{mathtools}
\usepackage{tikz}
\usepackage{graphicx}
\renewcommand{\thesubsection}{\alph{subsection})}
\title{Wintersemester 2015/2016 Inoffizielle Lösung}
\begin{document}
\section{}
\subsection{}
Carrier Sense Multiple Access / Collission Detection
\subsection{}
Konkurrierendes Zugriffsverfahren, da keine explizite Freigabe des Mediums erfordert wird.
\subsection{}
Nein, da irgendwann ein Maximum an Sendeversuchen erreicht wird, wonach einfach abgebrochen wird.
\subsection{}
Bietet: CRC (cyclic redundancy check)\\
Fehlt: Quittungen, Sequenznummern
\subsection{}
\subsection{}
$10110101 \oplus 11011000 = 01101101 = 5$\\
Hamming-Abstand: 5\\
Korregierbar sind nur 1 Bit-Fehler.
\subsection{}
Bündelfehler
\subsection{}
CRC hat keine Abhängigkeit zu irgendeinem geheimen Schlüssel\\ 
$\Rightarrow$ der Angreifer kann eine andere Nachricht erstellen und selber die Prüfsumme berechnen
\subsection{}
Systemneustart $\Rightarrow$ Sender und Empfänger hätte zum Beginn die gleiche Sequenznummer
\pagebreak
\section{}
\subsection{}
$Zeit =\frac{s}{v} = \frac{10km}{90km/h} = 400s$\\\\
$Datenrate = \frac{Mediengroeße}{Zeit} = \frac{4 TByte}{400s} = 10 GByte/s = 80 Gbit/s$
\subsection{}
Drohne: 4 TByte (die Festplatte die sie trägt) \\\\
$Puffergroeße = \frac{Abstand}{Ausbreitungsgeschwindigkeit} * Datenrate = \frac{10km}{200.000 km/s} * 40 Gbit/s
= 5 * 10^-5 s * 40 Gbit/s = 2 Mbit*s $
\subsection{}
$Laenge = \frac{Abstand}{Puffergroeße} = \frac{10km}{2Mbit} = 0.005 m/b$
\subsection{}
% nicht sicher
% $Datenrate = 2* Bandbreite * \log_2 Signalwerte pro Signalschritt = 2 * 500MHz * \log_2 4096 = 2 * 500MHz * 12 = 12 Gbit/s$
\subsection{}
Raummultiplexen scheidet aus, dieses wird zum Beispiel durch die Zellen beim Mobilfunk angewandt.
\pagebreak
\section{}
\begin{center}
\begin{tikzpicture}[auto, node distance=3.5cm, every loop/.style={},
                    thick,main node/.style={circle,draw,font=\sffamily\Large\bfseries}]

  \node[main node] (1) {A};
  \node[main node] (2) [right of=1] {B};
  \node[main node] (3) [below of=2] {C};
  \node[main node] (4) [below of=1] {D};

  \path[every node/.style={font=\sffamily\small}]
    (1) edge node [left] {8} (4)
        edge node {6} (2)
    (2) edge node {2} (4)
    (3) edge node {9} (2) 
    (4) edge node {1} (3);
\end{tikzpicture}
\end{center}
\subsection{} 
\paragraph{$t_0:$}
\begin{center}
  \begin{tabular}{|c|c|c|}
    \hline
    $D^A$ & B & D \\ \hline
    B & 6 & $\infty$ \\ \hline
    C & $\infty$ & $\infty$ \\ \hline
    D & $\infty$ & 8 \\
    \hline
  \end{tabular}
  \begin{tabular}{|c|c|c|c|}
    \hline
    $D^B$ & A & C & D \\ \hline
    A & 6 & $\infty$ & $\infty$ \\ \hline
    C & $\infty$ &9 & $\infty$ \\ \hline
    D & $\infty$ & $\infty$ & 2 \\
    \hline
  \end{tabular}
  \begin{tabular}{|c|c|c|}
    \hline
    $D^C$ & B & D \\ \hline
    A & $\infty$ & $\infty$ \\ \hline
    B & 9 & $\infty$ \\ \hline
    D & $\infty$ & 1 \\
    \hline
  \end{tabular}
  \begin{tabular}{|c|c|c|c|}
    \hline
    $D^D$ & A & B & C \\ \hline
    A & 8 & $\infty$ & $\infty$ \\ \hline
    B & $\infty$ &2 & $\infty$ \\ \hline
    C & $\infty$ & $\infty$ & 1 \\
    \hline
  \end{tabular}
\end{center}
\paragraph{$t_2:$}
\begin{center}
  \begin{tabular}{|c|c|c|}
    \hline
    $D^A$ & B & D \\ \hline
    B & 6 & 10 \\ \hline
    C & 15 & 9 \\ \hline
    D & 8 & 8 \\
    \hline
  \end{tabular}
  \begin{tabular}{|c|c|c|c|}
    \hline
    $D^B$ & A & C & D \\ \hline
    A & 6 &  & 10\\ \hline
    C &  &9 & 3 \\ \hline
    D & 14 & 10 & 2 \\
    \hline
  \end{tabular} \begin{tabular}{|c|c|c|}
    \hline
    $D^C$ & B & D \\ \hline
    A & 15 & 9  \\ \hline
    B & 9 &  11\\ \hline
    D & 11 & 1 \\
    \hline
  \end{tabular}
  \begin{tabular}{|c|c|c|c|}
    \hline
    $D^D$ & A & B & C \\ \hline
    A & 8 & 8 &  \\ \hline
    B & 14 &2 &10 \\ \hline
    C &  & 11 & 1 \\
    \hline
  \end{tabular}
\end{center}
\paragraph{$t_4:$}
\begin{center}
  \begin{tabular}{|c|c|c|}
    \hline
    $D^A$ & B & D \\ \hline
    B & 6 & 10 \\ \hline
    C & 9 &  9\\ \hline
    D & 8 &8  \\
    \hline
  \end{tabular}
  \begin{tabular}{|c|c|c|c|}
    \hline
    $D^B$ & A & C & D \\ \hline
    A & 6 & 12 & 10  \\ \hline
    C & 15 &9 &  3\\ \hline
    D &  14&  4& 2 \\
    \hline
  \end{tabular}
  \begin{tabular}{|c|c|c|}
    \hline
    $D^C$ & B & D \\ \hline
    A &15 & 9 \\ \hline
    B & 9 & 3\\ \hline
    D &  12& 1 \\
    \hline
  \end{tabular}
  \begin{tabular}{|c|c|c|c|}
    \hline
    $D^D$ & A & B & C \\ \hline
    A & 8 & 8 & 10 \\ \hline
    B & 14 &2 & 10 \\ \hline
    C &  17 & 5 & 1 \\
    \hline
  \end{tabular}
\end{center}

\subsection{}
Das Count-to-Infinity-Problem, wenn Linkkosten sich ändern brauch es relativ lang bis negative Änderungen sich ausbreiten.
Poisoned Reverse löst dies dadurch, dass ein Router X Routing-Informationen über einen Nachbar Y ``zurückhält'', wenn der Weg von ihm über einen anderen Nachbarn Z kürzer ist.
\subsection{}
\subsection{}
129.13.40.10 (kit.edu)
\subsection{}
Die Zuordnung von TCP- und UDP-Verbindungen und -Datenpaketen zu Server- und Client-Programmen durch Betriebssysteme.
\subsection{}
\begin{enumerate}
\item Erkennung von Duplikaten durch Sequenznummern
\item Kontrolle der Lebensdauer eines Datagramms durch Zählen der zurückgelegten Übertragungsabschnitte (Hops)
\begin{enumerate}
\item Hopzähler wird mit der maximalen Weglänge initialisiert
\item In jedem Router wird der Zähler um 1 dekrementiert
\item Falls der Zähler den Wert 0 erreicht, kann das Datagramm verworfen werden
\end{enumerate}
\end{enumerate}
Option 1 erfordert Sequenznummern im Header.
Option 2 einen modifizierbaren Hop-Zähler im Header.
\subsection{}
RIP (Routing Information Protocol), IGRP(Interior Gateway Routing Protocol), Babel
\subsection{}
Das Phänomen heißt Oszillation.
\begin{itemize}
\item Router berechnen den Algorithmus nicht zum selben Zeitpunkt
\item Hinzufügen eines Zufalls
\end{itemize}
\pagebreak
\section{}
\subsection{}
\subsection{}
\pagebreak
\section{Zusammenspiel der Schichten}
\subsection{}
Ja, eine Adresse wurde falsch gecached.
\pagebreak
\subsection{}
\rotatebox{90}{
\begin{tabular}{|c|c|c|c|c|c|c|c|}\hline
Zeile & Ethernet-Quelladresse & Ethernet-Zieladresse & Typ & IP-Quelladresse & IP-Zieladresse & Inhalt Anfrage & Inhalt Antwort \\ \hline
1 & MAC-A & MAC-BROAD & ARP & IP-A & IP-LDNS & - & - \\
2 & MAC-LDNS & MAC-A & ARP & IP-LDNS & IP-A & - & - \\
3 & MAC-A & MAC-LDNS & DNS & IP-A & IP-LDNS & www.berkeley.edu & - \\
4 & MAC-LDNS & MAC-LROUT & DNS & IP-LDNS & IP-ROOTDNS & .edu & - \\
5 & MAC-LROUT & MAC-LDNS & DNS & IP-ROOTDNS & IP-LDNS & - & IP-TLDDNS \\
6 & MAC-LDNS & MAC-LROUT & DNS & IP-LDNS & IP-TLDDNS & .berkeley.edu & - \\
7 & MAC-LROUT & MAC-LDNS & DNS & IP-TLDDNS & IP-LDNS & - & IP-BDNS \\
8 & MAC-LDNS & MAC-LROUT & DNS & IP-LDNS & IP-BDNS & www.berkeley.edu & - \\
9 & MAC-LROUT & MAC-LDNS & DNS & IP-BDNS & IP-LDNS & - & IP-WWW \\
10 & MAC-LDNS & MAC-A & DNS & IP-LDNS & IP-A & - & IP-WWW \\\hline
\end{tabular}
}
\end{document}
